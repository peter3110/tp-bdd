\documentclass[11pt, a4paper]{article}
\usepackage[paper=a4paper, left=1.5cm, right=1.5cm, bottom=1.5cm, top=1.5cm]{geometry}
%
\usepackage[utf8]{inputenc}
\usepackage[spanish]{babel}
\usepackage{ulem}

\usepackage{caratula/caratula}
\def\doubleunderline#1{\underline{\underline{#1}}}

\begin{document}

\titulo{Trabajo Práctico \#2 }
\fecha{5 de Noviembre de 2016}
\materia{Base de Datos}

\integrante{Pedro Rodriguez}{197/12}{pedro3110.jim@gmail.com}
\integrante{Lucas Tavolaro Ortiz}{322/12}{tavo92@gmail.com}
% \integrante{Pepe Sánchez}{444/12}{pepe@gmail.com}
% \integrante{Roberto Carlos}{111/10}{roberto@gmail.com}

%Carátula
\maketitle
\newpage

%Indice
\tableofcontents
\newpage

% Demás secciones
%
\section{Introducción}
% \input{introduccion.tex}

\section{Restricciones del lenguaje natural}
\begin{itemize}
	\item La fecha de cada testimonio no debe ser anterior a la fecha del caso en el cual este es presentado.
	\item La fecha de cada evento no debe ser anterior a la fecha del caso.
	\item La fecha de ingreso del investigador principal del caso no debe ser posterior a la fecha en que ocurrio el mismo.
	\item La fecha de ingreso de cada auxiliar del caso no debe ser posterior a la fecha en que ocurrio el mismo.
	\item La fecha de nacimiento de una persona no puede ser anterior a la fecha en la que ocurre el caso en el que esta involucrada.
	\item La fecha de ingreso de un oficial de policia no puede ser anterior a su fecha de nacimiento.
	\item Una evidencia no puede ser custodiada en dos localizaciones diferentes para una misma fecha y hora.
	\item Un departamento no puede supervisarse a si mismo.
	\item El culpable de un caso resuelto no puede ser una persona que no este involucrada en el caso.
	\item La fecha de congelado de un caso congelado no puede ser anterior a la fecha en que ocurrió el caso.
	\item La fecha de descarte de un caso descartado no puede ser anterior a la fecha en que ocurrió el caso.
	\item Si una persona es culpable de un Caso, o un Oficial resolvió un Caso, entonces el estado del caso es de tipo Resuelto.
\end{itemize}

\section{DER}
A continuación, mostramos el diagrama entidad relación utilizado como base para el desarrollo del TP
\includegraphics[scale=0.4]{der.png}

\section{MR}
A continuación, detallamos el modelo relacional correspondiente a nuestra BDD.

%% unitarios

\textbf{Persona} {\underline{dni}, nombre, apellido, fechaDeNacimiento, \
	\dashuline{idDomicilio}

\textbf{TelefonoPersona} {\underline{numero}, \dashuline{dniPersona}}

\textbf{TelefonoDepartamento} {\underline{numero}, \dashuline{idDepartamento}}

\textbf{RolEnCaso} {\underline{idRol}, nombreRol}

\textbf{CategoriaCaso} {\underline{idCategoria}, nombreCategoria, cantidadDeCasos}

\textbf{OficialDePolicia} {\underline{dni}, numeroPlaca, fechaIngreso, 
	numeroEscritorio, \dashuline{idRango}, \dashuline{idServicio},
	\dashuline{idDepartamento}}

\textbf{Servicio} {\underline{idServicio}, nombre}

\textbf{Rango} {\underline{idRango}, nombre}

\textbf{CasoCriminal} {\underline{idCaso}, fechaOcurrio, horaOcurrio, \ 
	lugarOcurrio, descripcion, fechaIngreso, \
	\dashuline{idCategoria}, \dashuline{idInvestigadorPrincipal}, \
	\dashuline{idInvestigadorResolvedor}}

\textbf{EstadoCaso} {\underline{idEstado}, \dashuline{idCaso}, \
	fecha}

\textbf{Resuelto} {\underline{idEstado}, \dashuline{idCaso}, \ 	
	descripcionResuelto}

\textbf{Pendiente} {\underline{idEstado}, \dashuline{idCaso},}

\textbf{Congelado} {\underline{idEstado}, \dashuline{idCaso}, \
	fechaCongelado, comentario}

\textbf{Descartado} {\underline{idEstado}, \dashuline{idCaso}, \
	fechaDescartado, motivo}

\textbf{Evento} {\underline{idEvento}, fechaOcurrio, horaOcurrio, descripcion} 

\textbf{Evidenca} {\underline{idEvidencia}, fechEncuentro, horaEncuentro, descripcion, \
	fechaSellado, horaSellado, fechaIngreso, \dashuline{idCaso}, }

\textbf{Departamento} {\underline{idDepartamento}, nombre, \dashuline{ \ 
	idDepartamentoSupervisor}, \dashuline{idDomicilio}}

\textbf{Domicilio} {\underline{idDomicilio}, altura, piso, depto, \
	\dashuline{idCalle}}

\textbf{Calle} {\underline{idCalle}, nombreCalle, \dashuline{idCiudad}}

\textbf{Ciudad} {\underline{idCiudad}, nombreCiudad, \dashuline{idProvincia}}

\textbf{Provincia} {\underline{idProvincia}, nombreProvincia}


%% relacions n:m
\textbf{Culpable} {\underline{dni}, \underline{idCaso}}

\textbf{InvestigadorAuxiliar} {\underline{idCaso}, \underline{dniOficialPolicia}}

%% n:m:p
\textbf{Involucra} {\underline{dni}, \underline{idCaso}, \dashuline{ \ 
	idRol}}

\textbf{ParticipaEn} {\underline{dni}, \underline{idEvento}, \dashuline{ \
	idCaso}}

\textbf{PresentaTestimonio} {texto, \underline{fecha}, \
	\underline{dniTestigo}, \
	\underline{idCaso}, \dashuline{dniOficial}}

\textbf{Custodia} {\underline{idEvidencia}, \underline{dniOficial}, \
	\dashuline{idDomicilio}, comentario, hora, fecha, \dashuline{idCustodia}}

%%%%%%%%%%%%%%%%%%%%%%

\section{Aclaraciones sobre las búsquedas}
Las búsquedas solicitas en el enunciado se encuentran en el archivo adjunto busquedas.sql

%
%
% \section{Conclusiones}
% \input{conclusiones.tex}
%

\end{document}

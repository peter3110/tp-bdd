\documentclass[11pt, a4paper]{article}
\usepackage[paper=a4paper, left=1.5cm, right=1.5cm, bottom=1.5cm, top=1.5cm]{geometry}
%
\usepackage[utf8]{inputenc}
\usepackage[spanish]{babel}
\usepackage{ulem}

\usepackage{caratula/caratula}
\def\doubleunderline#1{\underline{\underline{#1}}}

\begin{document}

\titulo{Trabajo Práctico \#2 }
\fecha{5 de Noviembre de 2016}
\materia{Base de Datos}

\integrante{Pedro Rodriguez}{197/12}{pedro3110.jim@gmail.com}
\integrante{Lucas Tavolaro Ortiz}{322/12}{tavo92@gmail.com}
% \integrante{Pepe Sánchez}{444/12}{pepe@gmail.com}
% \integrante{Roberto Carlos}{111/10}{roberto@gmail.com}

%Carátula
\maketitle
\newpage

%Indice
\tableofcontents
\newpage

% Demás secciones
%
\section{Introducción}
% \input{introduccion.tex}

\section{Suposiciones adicionales}
\begin{itemize}
	\item a
\end{itemize}

\section{Restricciones lenguaje natural}
\begin{itemize}
	\item a
\end{itemize}

\section{MR}
A continuación, detallamos el modelo relacional correspondiente a nuestra BDD.

%% unitarios

\textbf{Persona} {\underline{dni}, nombre, apellido, fechaDeNacimiento, \
	\dashuline{calleYnro}, \dashuline{provincia}, \dashuline{ciudad}}

\textbf{Telefono} {\underline{numero}, \dashuline{dni}, \dashuline{idDepartamento}}

\textbf{RolEnCaso} {\underline{idRol}, nombre}

\textbf{CategoriaCaso} {\underline{idCategoria}, nombre, cantidadDeCasos}

\textbf{OficialDePolicia} {\underline{dni}, numeroPlaca, fechaIngreso, 
	numeroEscritorio, \dashuline{idRango}, \dashuline{idServicio}}

\textbf{Servicio} {\underline{idServicio}, nombre}

\textbf{Rango} {\underline{idRango}, nombre}

\textbf{CasoCriminal} {\underline{idCaso}, fechaOcurrio, horaOcurrio, \ 
	lugarOcurrio, descripcion, fechaIngreso, \
	\dashuline{idCategoria}, \dashuline{idInvestigadorPrincipal}, \
	\dashuline{idInvestigadorResolvedor}}

\textbf{EstadoCaso} {\underline{idEstado}, \dashuline{idCaso}}

\textbf{Resuelto} {\underline{idEstado}, \ 	
	descripcionResuelto}

\textbf{Pendiente} {\underline{idEstado}}

\textbf{Congelado} {\underline{idEstado}, \
	fechaCongelado, comentario}

\textbf{Descartado} {\underline{idEstado}, \
	fechaDescartado, motivo}

\textbf{Evento} {\underline{idEvento}, fechaOcurrio, horaOcurrio, descripcion} 

\textbf{Evidenca} {\underline{idEvidencia}, fechEncuentro, horaEncuentro, descripcion, \
	fechaSellado, horaSellado, fechaIngreso, \dashuline{idCaso}, }

\textbf{Departamento} {\underline{idDepartamento}, nombre, \dashuline{ \ 
	idDepartamentoSupervisado}, \dashuline{calleYnro}, \dashuline{provincia}, \ 
	\dashuline{ciudad}}

\textbf{Domicilio} {\underline{calleYnro}, \underline{provincia}, \underline{ciudad}}

%% relacions n:m
\textbf{Culpable} {\underline{dni}, \underline{idCaso}}

\textbf{InvestigadorAuxiliar} {\underline{idCaso}, \underline{dniOficialPolicia}}

%% n:m:p
\textbf{Involucra} {\underline{dni}, \underline{idCaso}, \dashuline{ \ 
	idRol}}

\textbf{ParticipaEn} {\underline{dni}, \underline{idEvento}, \dashuline{ \
	idCaso}}

\textbf{PresentaTestimonio} {texto, hora, fecha, \underline{dniTestigo}, \
	\underline{idCaso}, \dashuline{dniOficial}}

\textbf{Custodia} {\underline{idEvidencia}, \underline{dniOficial}, \
	\dashuline{calleYnro}, \dashuline{provincia}, \dashuline{ciudad}}



\section{Aclaraciones sobre las búsquedas}
Las búsquedas solicitas en el enunciado se encuentran en el archivo adjunto busquedas.sql

%
%
% \section{Conclusiones}
% \input{conclusiones.tex}
%

\end{document}
